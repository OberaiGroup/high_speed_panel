\documentclass[conf]{new-aiaa}

\usepackage{float}
\usepackage{subfigure}
\usepackage[justification=centering]{caption}
\usepackage{multirow} \usepackage{graphicx}
\graphicspath{ {./images/} }
\usepackage{nameref}
\usepackage{amsmath}
\usepackage{amssymb}
\usepackage{amsfonts}
\usepackage[linesnumbered,ruled]{algorithm2e}
\usepackage{tikz}
\usetikzlibrary{calc,patterns,decorations.pathmorphing,decorations.markings,positioning,automata,shapes,arrows}
\tikzstyle{block} = [rectangle, draw, text width=3.5cm]
\tikzstyle{line} = [draw, -latex']
\usepackage{pgfplots}
\usepgfplotslibrary{colormaps,external}
\usepackage{ulem}
\usepackage{verbatim}
\usepackage[version=4]{mhchem}
\usepackage{siunitx}
\usepackage[super]{nth}
\usepackage{pbox}
\usepackage{longtable,tabularx}
\setlength\LTleft{0pt} 

\begin{document}

Consider a bar of length $L$ fixed on both ends.
It starts at temperature $\theta_0$ at a stress free state. 
A uniform temperature $\theta(t)$ is prescribed everywhere.
The total strain is:

\begin{equation}
\epsilon^T = \epsilon^e + \epsilon^c
\end{equation}

\noindent
with $\epsilon^T$ as the total strain,
$\epsilon^e$ as the elastic strain,
and
$\epsilon^c$ as the creep strain.
The total deflection (and strain) of the bar must be zero as it 
is fixed on both ends:

\begin{align}
\epsilon^T &= \epsilon^e + \epsilon^c = 0 \\
\Rightarrow
  \epsilon^e &= -\epsilon^c 
\end{align}

The stress from temperature change is:

\begin{equation}
\sigma = E \left( \epsilon^e + \alpha (\theta(t) - \theta_0) \right)
\end{equation}

\noindent
with $\sigma$ as the Cauchy stress, 
and 
$E$ as Young's modulus of the material.
Substituting in for the strain relation gives:

\begin{equation}
\sigma = E \left( \alpha (\theta(t) - \theta_0) - \epsilon^c) \right) 
\end{equation}

Assuming there is no volume change such that:

\begin{align}
\tau &= J \sigma \\
\Rightarrow
  \tau &= \sigma
\end{align}

\noindent
where $\tau$ is the Kirchoff stress,
$J=\det(F)$,
and $F$ is the deformation gradient.

The creep strain rate is:

\begin{equation}
\dot{\epsilon^c} = \bar{A} e^{\frac{-Q}{R \theta}} \left( \frac{\tau}{\sigma_0} \right)^{c_1}
\end{equation}

\noindent
with $\bar{A}$ as the relaxation rate,
$Q$ as the activation energy,
$R$ is the universal gas constant,
$\sigma_0$ is a reference stress,
and
$c_1$ is the stress exponent.
Plugging in for the Kirchoff stress:

\begin{equation}
\dot{\epsilon^c} = \bar{A} e^{\frac{-Q}{R \theta(t)}} \left( \frac{E ( \alpha \theta(t) - \epsilon^c)}{\sigma_0} \right)^{c_1}
\end{equation}

The is the ODE of the creep strain driven by the temperature change.

\end{document}
