\documentclass[conf]{new-aiaa}

\usepackage{float}
\usepackage{subfigure}
\usepackage[justification=centering]{caption}
\usepackage{multirow} \usepackage{graphicx}
\graphicspath{ {./images/} }
\usepackage{nameref}
\usepackage{amsmath}
\usepackage{amssymb}
\usepackage{amsfonts}
\usepackage[linesnumbered,ruled]{algorithm2e}
\usepackage{tikz}
\usetikzlibrary{calc,patterns,decorations.pathmorphing,decorations.markings,positioning,automata,shapes,arrows}
\tikzstyle{block} = [rectangle, draw, text width=3.5cm]
\tikzstyle{line} = [draw, -latex']
\usepackage{pgfplots}
\usepgfplotslibrary{colormaps,external}
\usepackage{ulem}
\usepackage{verbatim}
\usepackage[version=4]{mhchem}
\usepackage{siunitx}
\usepackage[super]{nth}
\usepackage{pbox}
\usepackage{longtable,tabularx}
\setlength\LTleft{0pt} 

\begin{document}

Consider a bar of length $L$ fixed on both ends.
It starts at temperature $\theta_0$ at a stress free state. 
A uniform temperature $\theta(t)$ is prescribed everywhere.
The total strain is:

\begin{equation}
\epsilon^T = \epsilon^e + \epsilon^c
\end{equation}

\noindent
with $\epsilon^T$ as the total strain,
$\epsilon^e$ as the elastic strain,
and
$\epsilon^c$ as the creep strain.
The total deflection (and strain) of the bar must be zero as it 
is fixed on both ends:

\begin{align}
\epsilon^T &= \epsilon^e + \epsilon^c = 0 \\
\Rightarrow
  \epsilon^e &= -\epsilon^c 
\end{align}

The stress from temperature change is:

\begin{equation}
\sigma = E \left( \epsilon^e + \alpha (\theta(t) - \theta_0) \right)
\end{equation}

\noindent
with $\sigma$ as the Cauchy stress, 
$\alpha$ as the Coefficient of Thermal Expansion (CTE),
and 
$E$ as Young's modulus of the material.
Substituting in for the strain relation gives:

\begin{equation}
\sigma = E \left( \alpha (\theta(t) - \theta_0) - \epsilon^c) \right) 
\end{equation}

Assuming there is no volume change such that:

\begin{align}
\tau &= J \sigma \\
\Rightarrow
  \tau &= \sigma
\end{align}

\noindent
where $\tau$ is the Kirchoff stress,
$J=\det(F)$,
and $F$ is the deformation gradient.

The creep strain rate is:

\begin{equation}
\dot{\epsilon^c} = \bar{A} e^{\frac{-Q}{R \theta}} \left( \frac{\tau}{\sigma_0} \right)^{c_1}
\end{equation}

\noindent
with $\bar{A}$ as the relaxation rate,
$Q$ as the activation energy,
$R$ is the universal gas constant,
$\sigma_0$ is a reference stress,
and
$c_1$ is the stress exponent.
Plugging in for the Kirchoff stress:

\begin{equation} \label{eq_creep_ode}
\dot{\epsilon^c} = \bar{A} e^{\frac{-Q}{R \theta(t)}} \left( \frac{E ( \alpha (\theta(t) - \theta_0) - \epsilon^c)}{\sigma_0} \right)^{c_1}
\end{equation}

This is the ODE of the creep strain driven by the temperature change.

Material properties for titanium alloy Ti6Al4V are then estimated from
Lavina et al. \cite{ lavina_creep_behavior_of_Ti6Al4V_from_450C_to_600C}
for creep parameters and from \cite{
boyer_materials_properties_handbook_titanium_alloys}
for the Young's modulus and the CTE of the material:

\begin{center}
\begin{tabular}{|c|c|c|c|}
\hline
Variable & Value & Units & Reference \\
\hline
$\sigma_0$ & 18.7  & MPa   & \cite{lavina_creep_behavior_of_Ti6Al4V_from_450C_to_600C} \\
$Q$      & 251     & kJ/Mol& \cite{lavina_creep_behavior_of_Ti6Al4V_from_450C_to_600C} \\
$c_1$    & 12.5    & 1     & \cite{lavina_creep_behavior_of_Ti6Al4V_from_450C_to_600C} \\
$E$      & 113.8   & GPa   & \cite{ boyer_materials_properties_handbook_titanium_alloys} \\
$\alpha$ & 9.7E-6  & 1/K   & \cite{ boyer_materials_properties_handbook_titanium_alloys} \\
\hline
\end{tabular}
\end{center}

For this unit set the universal gas constant is 8.3145 J/(Mol$\cdot$K)
and the relaxation parameter, $\bar{A}$, was taken as 1 with units of per second.

The initial and prescribed temperatures are borrowed from 
\cite{ culler_impact_of_FTS_coupling_on_response_prediction_hypersonic_skin_panels}.
In this study, the transient thermo mechanical response of a panel from the 
surface of a representative concept vehicle is simulated with a coupled framework.
The initial temperature is 70 $^{\circ}$F and a steady state temperature 
of 2750 $^{\circ}$F is achieved after the first 25 seconds of flight.
Note that the vehicle is assumed to start at its cruising speed of 12 Ma.
Incorporating into the bar problem (after converting to Kelvin) 
gives $\theta_0 = 294 K$ and:

\begin{equation}
\theta(t) = \begin{cases}
 \theta_0 + (59.65 \text{K/s})t & t < t_{ss} \\
1783 K                          & t\geq t_{ss}
\end{cases}
\end{equation}

\noindent
with $t_{ss}$ as the steady state time of 25 seconds.
This equates to a linear increase in temperature from 
$\theta_0$ at zero seconds to 1783 K at $t_{ss}$.
This temperature is then held constant.
For reference, a measure of the ``activation temperature'' 
$\left(\frac{Q}{R}\right)$ is 3019 K.

The ODE in Eq \ref{eq_creep_ode} was solved over a time period to represent a
vehicle's total flight time.
Based on the X-15, flights with an average of 10 minutes each 
\cite{ kordes_structureal_heating_experiencs_on_the_x15_airplane}
were considered.
Only one flight was considered giving a total time of 600 seconds.
The creep strain during this single flight is shown in Figure
\ref{fig_creep_strain}.
A focused view on the first 50 seconds of flight is shown 
in Figure \ref{fig_focused_creep_strain}.
Note the estimated thermal strain is 1.4\%.

\begin{figure}[H]
  \centering
    \begin{tikzpicture}
      \begin{axis}[
        legend pos=outer north east,
        legend cell align={left},
        grid=both,
        grid style={line width=.1pt, draw=gray!10},
        major grid style={line width=.2pt,draw=gray!50},
        xtick={},
        ytick={},
        minor tick num=5,
        xlabel=Time (s),
        title=Creep Strain from One Flight,
        ylabel={Creep Strain (\%)} 
        ]
        \addplot table {data/solution.txt};
      \end{axis}
    \end{tikzpicture}
  \caption{ Creep strain over time.}
  \label{fig_creep_strain}
\end{figure}

\begin{figure}[H]
  \centering
    \begin{tikzpicture}
      \begin{axis}[
        legend pos=outer north east,
        legend cell align={left},
        grid=both,
        grid style={line width=.1pt, draw=gray!10},
        major grid style={line width=.2pt,draw=gray!50},
        xtick={},
        ytick={},
        minor tick num=5,
        xlabel=Time (s),
        title=Creep Strain over First 50 Seconds of Flight,
        ylabel={Creep Strain (\%) } 
        ]
        \addplot table {data/focused_solution.txt};
      \end{axis}
    \end{tikzpicture}
  \caption{ Creep strain over time.}
  \label{fig_focused_creep_strain}
\end{figure}

As shown in Figures \ref{fig_creep_strain}
and \ref{fig_focused_creep_strain}, the creep strain in the bar 
grows to match that from the thermal strain. 
The creep strain reaches an approximate steady state 
at 25 seconds into the flight. 
This matches the time when the temperature also reaches
its stead state value.


\newpage
\bibliography{monolith.bib}

\end{document}

