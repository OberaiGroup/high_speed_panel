\documentclass[conf]{new-aiaa}

\usepackage{float}
\usepackage{subfigure}
\usepackage[justification=centering]{caption}
\usepackage{multirow} \usepackage{graphicx}
  \graphicspath{ {./images/} }
\usepackage{nameref}
\usepackage{amsmath}
\usepackage{amssymb}
\usepackage{amsfonts}
\usepackage[linesnumbered,ruled]{algorithm2e}
\usepackage{tikz}
\usetikzlibrary{calc,patterns,decorations.pathmorphing,decorations.markings,positioning,automata,shapes,arrows}
  \tikzstyle{block} = [rectangle, draw, text width=3.5cm]
  \tikzstyle{line} = [draw, -latex']
\usepackage{pgfplots}
  \pgfplotsset{compat=1.3}
\usepgfplotslibrary{colormaps,external}
\usepackage{ulem}
\usepackage{verbatim}
\usepackage[version=4]{mhchem}
\usepackage{siunitx}
\usepackage[super]{nth}
\usepackage{pbox}
\usepackage{longtable,tabularx}
  \setlength\LTleft{0pt} 

\title{ Deformation of a Panel in Repeated High Speed Flow 
        Modeled with Creep and Multiplicatively-Decomposed Plasticity}

\author{Justin L. Clough%
        \footnote{
          PhD Student, 
          justin.clough1@gmail.com,
          Aerospace and Mechanical Engineering Department, 
          854 Downey Way, RRB 101,
          Student Member}
        and Assad A. Oberai%
        \footnote{  
          Professor, 
          Aerospace and Mechanical Engineering Department, 
          854 Downey Way, RRB 101.}}
\affil{University of Southern California,
       Los Angeles, CA, 90089}

\author{Jos\'e A. Camberos%
        \footnote{
          Team Lead of Design of Aerospace Systems for Hypersonics, 
          Air Force Research Laboratory,
          Wright-Patterson AFB, OH 45433,
          Associate Fellow.
          \newline
          \newline
          DISTRIBUTION STATEMENT TEXT}} 
\affil{Air Force Research Laboratory, Wright-Patterson AFB,
       Dayton, OH, 45433}


\begin{document}
\maketitle

\begin{abstract}
The surface skin panels of high speed vehicles can reach high 
temperatures when in flight. 
These temperatures are greater than the solidus temperature
of aluminum alloys and in the range where
creep has been observed for titanium alloys.
Studies have modeled the permanent set and deformation 
of panels using small strain plasticity.
They have also accounted for temperature
dependent material properties in their constitutive models.
This work models the creep of a stiffened panel
from a conceptual high speed vehicle.
A titanium alloy, Ti6Al4V,
is used as it is common in the design of high speed vehicles.
Multiplicatively-decomposed plasticity 
is included in the constitutive law as well to allow for large strains.
The panel geometry from a high speed concept vehicle will be
used with boundary conditions to induce thermal buckling
as opposed to flutter.
The temperature of the panel is prescribed uniformly 
and based on recent literature estimates.
The temperature is cycled to simulate phases of
heating, cruise, and cooling over multiple flights.
Preliminary analytical work considered a single strip
of vehicle paneling under two flight cycles.
The results show that the creep strain
grew to within 98\% of those induced by thermal strains 
over two representative flight cycles.
\end{abstract}

\section{Nomenclature}

{\renewcommand\arraystretch{1.0}
\noindent\begin{longtable*}{@{}l @{\quad=\quad} l@{}}
$\bar{A}$    & Relaxation rate\\
$c_1$        & Stress Exponent\\
$E$          & Young's Modulus \\
$F$          & Deformation gradient\\
$J$          & Ratio of volumetric change\\
Ma           & Mach number\\
$Q$          & Activation energy\\
$R$          & Universal gas constant\\
$t$          & Time\\
$t_{ss}$     & Steady state time\\
$t_{f}$      & Flight time \\
$\alpha$     & Coefficient of thermal expansion\\
$\epsilon^T$ & Total strain\\
$\epsilon^e$ & Elastic strain\\
$\epsilon^c$ & Creep strain\\
$\sigma$     & Cauchy stress\\
$\sigma_0$   & Reference stress\\
$\tau$       & Kirchhoff stress\\
$\theta$     & Temperature\\
$\theta_0$   & Initial temperature
\end{longtable*}}

\section{Introduction}
% Outline:
% - Overview of hypersonic structures
%   - Emph: re-usability
% - What is high speed flow? Characteristics
% - When does and doesn't inertia matter?
% - When does large strain matter?
% - What is creep/ Norton creep?

High speed flow over a vehicle creates a harsh environment for 
the vehicle structure.
For example, the skin of the X-15 regularly saw temperatures
as high as 1220 $^{\circ}$F (933 K)
during its flights up to Mach 6.7
\cite{ kordes_structureal_heating_experiencs_on_the_x15_airplane}.
A recent study by Culler et al.
\cite{ culler_impact_of_FTS_coupling_on_response_prediction_hypersonic_skin_panels}
estimated temperatures exceeding 2500 $^\circ$F (1644 K) 
on the skin of a conceptual high speed vehicle.
These temperatures are greater than the solidus 
temperature of typical aluminum alloys used in aircraft design
\cite{ SAE_metals_and_alloys_in_the_unified_numbering_system}
and within the range where creep has been observed for 
high temperature titanium alloys 
\cite{
  evans_effects_of_alpha_case_formation_on_creep_fracture_properties_of_the_high_temperature_titanium_alloy_IMI834,
  lavina_creep_behavior_of_Ti6Al4V_from_450C_to_600C}.

The skin of a high speed vehicle must support both
the aerodynamic tractions and heating during flight.
Recent interest in reusable vehicles requires
the skin to endure multiple cycles of this loading
\cite{
  walker_falcon_htv_3X_a_resuable_hypersonic_test_bed,
  eason_structures_perspective_on_the_challenges_associated_with_analyzing_reuasble_hypersonic_platform,
  zuchowski_AVIATR_Predictive_capability_for_hypersonic_structural_response_and_life_prediction_phase_II,
  lafontaine_effects_of_strain_hardeing_on_response_of_skin_panels_in_hypersonic_flow}.
The skin panels are typically designed as thin face plates supported 
by stiffeners 
\cite{
  mcnamara_aeroelastic_and_aerothermoelastic_analysis_in_hypersonic_flow_past_present_and_future}.
The face plate and stiffeners of the panel 
can be made of materials on the order
of hundredth of an inch thick
\cite{
  plews_a_two_scale_generalized_finite_element_approach_for_modeling_localized_thermoplasticity,
  zuchowski_AVIATR_Predictive_capability_for_hypersonic_structural_response_and_life_prediction_phase_II}.
The panels are compliant and can exhibit combinations of
inertia dominate flapping or thermally dominate buckling
\cite{
  thornton_coupled_flow_thermal_and_structural_analysis_of_aerodynamically_heated_panels,
  mei_review_of_nonlinear_panel_flutter_at_supersonic_and_hypersonic_speeds}.
Much work has been done to investigate the fluttering 
of panels in high speed flow
\cite{
  mcnamara_aeroelastic_and_aerothermoelastic_analysis_in_hypersonic_flow_past_present_and_future,
  riley_interaction_between_aerothermally_compliant_structures_and_boudnary_layer_transition_in_hypersonic_flow,
  spottswood_exploring_the_response_of_a_thin_flexible_panel,
  savino_aerothermodynamic_study_of_ultrahigh_termperature_cermaic_winglet_for_atmospheric_reentry_test,
  nydick_hypersonic_panel_flutter_studies_on_cruved_panels,
  lafontaine_effects_of_strain_hardeing_on_response_of_skin_panels_in_hypersonic_flow}.


A study by LaFontaine et al.
\cite{
  lafontaine_effects_of_strain_hardeing_on_response_of_skin_panels_in_hypersonic_flow}
investigated the effects of accumulated 
plastic strain and permanent set of a panel in high speed flow.
The panel material in this study was modeled with an additive 
strain decomposition, a Chaboche kinematic hardening model,
and a $J_2$ flow rule. 
The panel interaction with the flow was modeled through
Eckert's reference enthalpy method 
\cite{
  eckert_engineering_relations_for_heat_transfer_and_friction_in_high_velocity_laminar_and_turbulent_flow}
and \nth{3} order piston theory
\cite{
  meijer_generalized_formulation_and_review_of_piston_theory_for_airfoils}.
The study found that by including plasticity, 
the behavior of the panel changed after four loading cycles 
as compared to an otherwise equivalent linearly elastic panel;
each cycle represented one ten second flight.
Both the elastic and plastic panel originally showed a mix of 
aeroelastic flutter with periods of thermal buckling. 
After the third cycle, the plastic panel demonstrated only thermal
buckling behavior.

This study further investigates the role creep and plasticity play 
in the permanent deformation of a panel under repeated high speed flight.
The amount of deformation of a body undergoes due to creep increases 
with both time and temperature
\cite{roylance_mechanics_of_materials_text}. 
Vehicle flight times are typically estimated on the order
of tens of minutes 
\cite{ 
  kordes_structureal_heating_experiencs_on_the_x15_airplane,
  lafontaine_effects_of_strain_hardeing_on_response_of_skin_panels_in_hypersonic_flow,
  zuchowski_AVIATR_Predictive_capability_for_hypersonic_structural_response_and_life_prediction_phase_II}
which is less than the timescale of hours that creep is 
typically associated with 
\cite{ 
  lavina_creep_behavior_of_Ti6Al4V_from_450C_to_600C,
  evans_effects_of_alpha_case_formation_on_creep_fracture_properties_of_the_high_temperature_titanium_alloy_IMI834,
  roylance_mechanics_of_materials_text}.
However, creep effects have been found to be significant on smaller timescales,
on the order of tens of minutes,
where large temperature fluctuations and gradients are found.
An example of this is in flip-chip manufacturing 
where silicon chips undergo a 150 K temperature change 
in 15 minutes
\cite{ 
  li_simulation_of_finite_strain_inelastic_phenomena_governed_by_creep_and_plasticity}.
Plasticity through a von Mises yield surface is also modeled;
a multiplicatively decomposed version is used to allow for 
possible large strains.
The details of the model and implementation used are
discussed in 
\cite{ li_simulation_of_finite_strain_inelastic_phenomena_governed_by_creep_and_plasticity}.
This extended abstract considers the creep strain of a one dimensional bar
representing a strip of vehicle paneling.
A uniform temperature is prescribed with heating rates and saturation 
temperature based on recent literature estimates.
\newline
\newline
\noindent
\emph{Note to the reviewers:} A more extensive review of the 
state of the art will be included in the 
final manuscript.

\section{Methods} \label{sec_methods}
% outline:
%  - Panel geometry used
%  - Explain Creep + Plasticity from Zhen's model
%  - Loading: Repeated historical temperature
%  - Compare results to literature

The goal of this study is to model the large creep deformation of a canonical acreage
panel due to the thermal loads experienced from repeated high speed flight.
As an approximation of the expected creep strain,
a bar that is fixed on both ends is considered as it 
undergoes a relevant temperature history.
This bar represents one strip of skin paneling from 
a high speed vehicle.
The bar starts at temperature $\theta_0$ at a stress free state. 
A uniform temperature $\theta(t)$ is prescribed everywhere.
The total strain within the bar is:

\begin{equation}
\epsilon^T = \epsilon^e  + \epsilon^\theta+ \epsilon^c
\label{eq_total_strain_sum}
\end{equation}

\noindent
with $\epsilon^T$ as the total strain,
$\epsilon^e$ as the elastic strain,
$\epsilon^\theta$ as the thermal strain,
and
$\epsilon^c$ as the creep strain.
The total deflection, and strain, of the bar must be zero as it 
is fixed on both ends.
Applying this constrain to Eq. \ref{eq_total_strain_sum} gives:

\begin{align}
\epsilon^T &= \epsilon^e + \epsilon^\theta + \epsilon^c = 0 \\
\Rightarrow
  \epsilon^e &= -\left( \epsilon^\theta + \epsilon^c \right)  \label{eq_strain_balance}
\end{align}

\noindent
The stress from temperature change is approximated by
a linear stress-temperature relation. 
The Cauchy stress is given as:

\begin{equation}
\sigma = E  \epsilon^e 
\end{equation}

\noindent
with $\sigma$ as the Cauchy stress and $E$ as Young's Modulus.
Substituting for the strain relation from Eq. \ref{eq_strain_balance} 
expanding the thermal strain gives:

\begin{align}
\sigma &= -E \left( \epsilon^\theta + \epsilon^c \right)  \\
\sigma &= -E \left( \alpha (\theta(t) - \theta_0) + \epsilon^c) \right) 
\label{eq_cauchy_stress_sum}
\end{align}

\noindent
where $\alpha$ is the Coefficient of Thermal Expansion (CTE).
Next, it is assumed that there is no volume change over time.
This is consistent with the fixed boundary conditions at 
each end of the bar.
This allows the Kirchhoff stress to be equivalent to the Cauchy stress:

\begin{align}
\tau &= J \sigma \\
\Rightarrow
  \tau &= \sigma \label{eq_cauchy_to_kirchoff}
\end{align}

\noindent
where $\tau$ is the Kirchhoff stress,
$J=\det(F)$,
and $F$ is the deformation gradient.
The creep strain rate, as outlined in 
\cite{ li_simulation_of_finite_strain_inelastic_phenomena_governed_by_creep_and_plasticity},
is:

\begin{equation}
\dot{\epsilon^c} = \bar{A} e^{\frac{-Q}{R \theta(t)}} \left( \frac{ ||\tau||}{\sigma_0} \right)^{c_1} \text{sgn}(\tau)
\end{equation}

\noindent
with $\bar{A}$ as the relaxation rate,
$Q$ as the activation energy,
$R$ is the universal gas constant,
$\sigma_0$ is a reference stress,
$c_1$ is the stress exponent,
and
$\text{sgn}(\cdot)$ as the sign operator.
Evaluating for the Kirchhoff stress as shown in Eq. \ref{eq_cauchy_stress_sum} and \ref{eq_cauchy_to_kirchoff}
then yields the following ODE.

\begin{equation} \label{eq_creep_ode}
\dot{\epsilon^c} = \bar{A} e^{\frac{-Q}{R \theta(t)}} 
    \left( \frac{E ( \alpha (\theta(t) - \theta_0) + \epsilon^c)}{\sigma_0} \right)^{c_1} 
    \text{sgn}\left( -E \left( \alpha (\theta(t) - \theta_0) + \epsilon^c) \right) \right)
\end{equation}

This is the ODE of the creep strain driven by the temperature change.
Material properties for titanium alloy Ti6Al4V were then estimated from
Lavina et al. \cite{ lavina_creep_behavior_of_Ti6Al4V_from_450C_to_600C}
for creep parameters and from \cite{
boyer_materials_properties_handbook_titanium_alloys}
for the Young's modulus and the CTE of the material.
For this unit set the universal gas constant is 8.3145 J/(Mol$\cdot$K)
and the relaxation parameter, $\bar{A}$, was taken as 1 with units of per second.
The material parameters used are shown in Table \ref{tab_material_properties}.

\begin{table}[H]
  \centering
  \caption{
    Material properties used for Ti6Al4V.
    Data from 
    \cite{ lavina_creep_behavior_of_Ti6Al4V_from_450C_to_600C,
      boyer_materials_properties_handbook_titanium_alloys}.}
  \begin{tabular}{|c|c|c|}
    \hline
    Variable & Value & Units  \\
    \hline
    $\sigma_0$ & 22.7  & MPa    \\
    $Q$      & 251     & kJ/Mol \\
    $c_1$    & 12.5    & 1      \\
    $E$      & 113.8   & GPa    \\
    $\alpha$ & 9.7E-6  & 1/K    \\
    \hline
  \end{tabular}
  \label{tab_material_properties}
\end{table}


The initial and prescribed temperatures are borrowed from 
\cite{ culler_impact_of_FTS_coupling_on_response_prediction_hypersonic_skin_panels}.
In this study, the transient thermo mechanical response of a panel from the 
surface of a representative concept vehicle is simulated with a coupled framework.
The initial temperature is 70 $^{\circ}$F (294 K) and a steady state temperature 
of 2750 $^{\circ}$F (1283 K) is achieved after the first 25 seconds of flight.
Note that the vehicle is assumed to start at its cruising speed of Mach 12
as done in \cite{ culler_impact_of_FTS_coupling_on_response_prediction_hypersonic_skin_panels}.
Incorporating into the bar problem
gives $\theta_0 = 294 K$ and:

\begin{equation}
\theta(t) = \begin{cases}
 \theta_0 + (59.65 \text{K/s})t                & t < t_{ss} \\
1783 K                                         & t_{ss} \leq t < t_f-t_{ss} \\
(1783 K) - (59.65 \text{K/s})(t-(t_f-t_{ss}))  & t_f - t_{ss} \leq t
\end{cases}
\end{equation}

\noindent
with $t_{ss}$ as the steady state time of 25 seconds
and $t_f$ representing one flight time.
It was assumed that the steady state time for heating was
equal to that of cooling.
This equates to a linear increase in temperature from 
$\theta_0$ at zero seconds to 1783 K at $t_{ss}$;
this temperature is then held constant until $t_{ss}$ before
the end of the flight.
The temperature then decreases linearly with time back to $\theta_0$.
This is equivalent to a set of heating, cruise, and cooling phases of a single flight.
For reference, a measure of the ``activation temperature'' 
$\left(\frac{Q}{R}\right)$ of Ti6Al4V is 3019 K.

The ODE in Eq \ref{eq_creep_ode} was solved over a time period to represent a
vehicle's total flight time.
Based on the X-15, flights with an average of 10 minutes each 
\cite{ kordes_structureal_heating_experiencs_on_the_x15_airplane}
were considered.
A repeated set of two flights were considered giving a total time of 1200 seconds.
Figure \ref{fig_creep_strain} shows
the creep strain during these back-to-back flights.
A focused view on the first 50 seconds of flight is shown 
in Figure \ref{fig_takeoff_creep_strain}.
A similar view of 250 seconds centered at the change in flights is 
shown in Figure \ref{fig_touchdown_creep_strain}.
The estimated thermal strain at the steady state temperature is 1.44\%.
The prescribed temperature is shown in Figure \ref{fig_full_temperature}.

% TODO:
% - Add plot of sigma/sigma_0
% - Add plot of thermal strain

\begin{figure}[H]
  \centering
    \begin{tikzpicture}
      \begin{axis}[
        legend pos=outer north east,
        legend cell align={left},
        grid=both,
        grid style={line width=.1pt, draw=gray!10},
        major grid style={line width=.2pt,draw=gray!50},
        xtick={},
        ytick={},
        minor tick num=5,
        xlabel=Time (s),
        title=Temperature Prescribed over Two Flights,
        ylabel={Temperature (K)} 
        ]
        \addplot[red,thick] table {data/temperature.txt};
      \end{axis}
    \end{tikzpicture}
  \caption{ Prescribed temperature over time of two flight cycles.}
  \label{fig_full_temperature}
\end{figure}

\begin{figure}[H]
  \centering
    \begin{tikzpicture}
      \begin{axis}[
        legend pos=outer north east,
        legend cell align={left},
        grid=both,
        grid style={line width=.1pt, draw=gray!10},
        major grid style={line width=.2pt,draw=gray!50},
        xtick={},
        ytick={},
        yticklabel style={/pgf/number format/.cd, fixed zerofill},
        minor tick num=5,
        xlabel=Time (s),
        title=Creep Strain from Repeated Flights,
        ylabel={Creep Strain (\%%
                )} 
        ]
        \addplot[black, thick] table {data/solution.txt};
      \end{axis}
    \end{tikzpicture}
  \caption{ Creep strain over time of two flight cycles of heating, cruise, and cooling.}
  \label{fig_creep_strain}
\end{figure}

\begin{figure}[H]
  \centering
    \begin{tikzpicture}
      \begin{axis}[
        legend pos=outer north east,
        legend cell align={left},
        grid=both,
        grid style={line width=.1pt, draw=gray!10},
        major grid style={line width=.2pt,draw=gray!50},
        xtick={},
        ytick={},
        yticklabel style={/pgf/number format/.cd, fixed zerofill},
        minor tick num=5,
        xlabel=Time (s),
        title=Creep Strain over First 50 Seconds of Flight,
        ylabel={Creep Strain (\%%
                  ) } 
        ]
        \addplot[black, thick] table {data/takeoff_solution.txt};
      \end{axis}
    \end{tikzpicture}
  \caption{ Creep strain during first thermal transient phase.}
  \label{fig_takeoff_creep_strain}
\end{figure}

\begin{figure}[H]
  \centering
    \begin{tikzpicture}
      \begin{axis}[
        legend pos=outer north east,
        legend cell align={left},
        grid=both,
        grid style={line width=.1pt, draw=gray!10},
        major grid style={line width=.2pt,draw=gray!50},
        xtick={},
        ytick={},
        yticklabel style={/pgf/number format/.cd, precision=5, zerofill},
        minor tick num=5,
        xlabel=Time (s),
        title=Creep Strain during Cooling and Re-Heating,
        ylabel={Creep Strain (\%%
                ) } 
        ]
        \addplot[black, thick] table {data/touchdown_solution.txt};
      \end{axis}
    \end{tikzpicture}
  \caption{ Creep strain during 250 seconds covering the cooling and reheating phases.}
  \label{fig_touchdown_creep_strain}
\end{figure}

As shown in Figures \ref{fig_creep_strain}
and \ref{fig_takeoff_creep_strain}, the creep strain in the bar 
grows to match that from the thermal strain. 
The creep strain reaches an approximate steady state 
at 25 seconds into the flight. 
This matches the time when the temperature also reaches
its steady state value.
The value of the creep strain does not start noticeable increasing
until approximately 7 seconds into the flight;
this corresponds to a temperature of 714 K
and thermal stress of 463.6 MPa.

The creep strain stagnates and decreases during the cooling 
and reheating phases, as shown in Figure \ref{fig_touchdown_creep_strain}.
Before cooling begins (575 s), the creep strain steadily increases to 
match that of the thermally induced strain.
During the cooling phase (575 s to 600 s), the creep strain
decreases slightly and stagnates. 
The initial decrease is due to the thermal contraction of the material.
The later stagnation due to the temperature dropping below the needed 
threshold to activate creep, essentially ``freezing'' the material.
The reheating phase (600 s to 625 s) first shows a decrease then rise in 
the amount of creep stain.
The decrease during reheating is due to the temperature surpassing the
needed temperature to activate creep but causing a thermal strain
lower than the creep strain.
The later rise of creep strain in the reheating phase is due
to the temperature being both great enough to activate creep
and cause a thermal strain greater than the creep strain.
After the reheating phase (625 s), the material behavior
returns to show a creep strain slowly approaching the steady 
thermal strain.

This preliminary work shows that the creep mechanism plays 
an active role in the deformation of high speed panels in flight.
Additional work to be completed includes the following items:

\begin{itemize}
  \item Create a geometric estimation and discreization of a high speed panel.
        A geometry based on the one used in 
        \cite{ culler_impact_of_FTS_coupling_on_response_prediction_hypersonic_skin_panels}
        will be used in this study.
  \item Evaluate the panel using the creep and plasticity model and implementaion presented in
        \cite{ li_simulation_of_finite_strain_inelastic_phenomena_governed_by_creep_and_plasticity}.
        A temperature history from 
        \cite{ culler_impact_of_FTS_coupling_on_response_prediction_hypersonic_skin_panels}
        will be prescribed throughout the panel. 
        Multiple flight cycles of heating, cruise, and cooling will be used
        to match historical and conceptual vehicles outlined in
        \cite{ kordes_structureal_heating_experiencs_on_the_x15_airplane,
               zuchowski_AVIATR_Predictive_capability_for_hypersonic_structural_response_and_life_prediction_phase_II}.
  \item Compare the deformation and permanent set of the panel to that 
        found by \cite{ culler_impact_of_FTS_coupling_on_response_prediction_hypersonic_skin_panels}.
\end{itemize}

\noindent
The post-analysis and results will include plots the maximum creep strain, norms of stress, 
and deflection throughout the panel and at points of interest over time. 
Contour plots will also be made showing the total strain, stresses, and deflections
of the panel at key points in time.
Following discussions include the importance of creep 
with respect to other permanent deformation mechanism 
in the high speed flight environment.


% \section{Discussion and Closing Remarks} \label{sec_closing_remarks}
% discussion text

% \section{Future Work} \label{sec_future}
% future work text
% 
% \section*{Acknowledgments}
% The authors thank the DoD Science, Mathematics, And 
% Research for Transformation (SMART) Scholarship
% for Service Program which sponsored this work.

\bibliography{monolith.bib}

\end{document}
