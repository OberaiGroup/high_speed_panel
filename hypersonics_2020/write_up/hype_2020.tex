\documentclass[conf]{new-aiaa}

\usepackage{float}
\usepackage{subfigure}
\usepackage[justification=centering]{caption}
\usepackage{multirow} \usepackage{graphicx}
\graphicspath{ {./images/} }
\usepackage{nameref}
\usepackage{amsmath}
\usepackage{amssymb}
\usepackage{amsfonts}
\usepackage[linesnumbered,ruled]{algorithm2e}
\usepackage{tikz}
\usetikzlibrary{calc,patterns,decorations.pathmorphing,decorations.markings,positioning,automata,shapes,arrows}
\tikzstyle{block} = [rectangle, draw, text width=3.5cm]
\tikzstyle{line} = [draw, -latex']
\usepackage{pgfplots}
\usepgfplotslibrary{colormaps,external}
\usepackage{ulem}
\usepackage{verbatim}
\usepackage[version=4]{mhchem}
\usepackage{siunitx}
\usepackage[super]{nth}
\usepackage{pbox}
\usepackage{longtable,tabularx}
\setlength\LTleft{0pt} 

\title{ Deformation Modeled with  Creep and Multiplicatively-Decomposed Plasticity
        of a Panel in Repeated High Speed Flow}
      

\author{Justin L. Clough%
        \footnote{
          PhD Student, 
          justin.clough1@gmail.com,
          Aerospace and Mechanical Engineering Department, 
          854 Downey Way, RRB 101.}
        and Assad A. Oberai%
        \footnote{  
          Professor, 
          Aerospace and Mechanical Engineering Department, 
          854 Downey Way, RRB 101.}}
\affil{University of Southern California,
       Los Angeles, CA, 90089}
\begin{document}
\maketitle

\begin{abstract}
abstract text
\end{abstract}

\section{Nomenclature}

{\renewcommand\arraystretch{1.0}
\noindent\begin{longtable*}{@{}l @{\quad=\quad} l@{}}
Ma & Mach number
\end{longtable*}}

\section{Introduction}
% Outline:
% - Overview of hypersonic structures
%   - Emph: re-usability
% - What is high speed flow? Characteristics
% - When does and doesn't inertia matter?
% - When does large strain matter?
% - What is creep/ Norton creep?

High speed flow over a vehicle creates a harsh environment for 
the vehicle structure.
For example, the skin of the X-15 regularly saw temperatures
as high as 1220$^{\circ}$ F
during its flights up to 6.7 Ma
\cite{ kordes_structureal_heating_experiencs_on_the_x15_airplane}.
This temperature is greater than the solidus 
temperature of typical aluminum alloys used in aircraft design
\cite{ SAE_metals_and_alloys_in_the_unified_numbering_system}
and within the range where creep has been observed for 
high temperature titanium alloys 
\cite{evans_effects_of_alpha_case_formation_on_creep_fracture_properties_of_the_high_temperature_titanium_alloy_IMI834}.


The skin of a high speed vehicle must support both
the aerodynamic tractions and heating during flight.
Recent interest in reusable vehicles requires
the skin to endure multiple cycles of this loading
\cite{
  walker_falcon_htv_3X_a_resuable_hypersonic_test_bed,
  eason_structures_perspective_on_the_challenges_associated_with_analyzing_reuasble_hypersonic_platform,
  zuchowski_AVIATR_Predictive_capability_for_hypersonic_structural_response_and_life_prediction_phase_II,
  lafontaine_effects_of_strain_hardeing_on_response_of_skin_panels_in_hypersonic_flow}.
The skin panels are typically designed as thin face plates supported 
by stiffeners 
\cite{
  mcnamara_aeroelastic_and_aerothermoelastic_analysis_in_hypersonic_flow_past_present_and_future}.
The face plate and stiffeners of the panel 
can be made of materials on the order
of hundredth of inch thick
\cite{
  plews_a_two_scale_generalized_finite_element_approach_for_modeling_localized_thermoplasticity,
  zuchowski_AVIATR_Predictive_capability_for_hypersonic_structural_response_and_life_prediction_phase_II}.
The panels are compliant and can exhibit combinations of
inertia dominate flapping or thermally dominate buckling
\cite{
  thornton_coupled_flow_thermal_and_structural_analysis_of_aerodynamically_heated_panels,
  mei_review_of_nonlinear_panel_flutter_at_supersonic_and_hypersonic_speeds}.
Much work has been done to investigate the fluttering 
of panels in high speed flow
\cite{
  mcnamara_aeroelastic_and_aerothermoelastic_analysis_in_hypersonic_flow_past_present_and_future,
  riley_interaction_between_aerothermally_compliant_structures_and_boudnary_layer_transition_in_hypersonic_flow,
  spottswood_exploring_the_response_of_a_thin_flexible_panel,
  savino_aerothermodynamic_study_of_ultrahigh_termperature_cermaic_winglet_for_atmospheric_reentry_test,
  nydick_hypersonic_panel_flutter_studies_on_cruved_panels,
  lafontaine_effects_of_strain_hardeing_on_response_of_skin_panels_in_hypersonic_flow}.


A study by LaFontaine et al.
\cite{
  lafontaine_effects_of_strain_hardeing_on_response_of_skin_panels_in_hypersonic_flow}
investigated the effects of accumulated 
plastic strain and permanent set of a panel in high speed flow.
The panel material in this study was modeled with an additive 
strain decomposition, a Chaboche kinematic hardening model,
and a $J_2$ flow rule. 
The panel interaction with the flow was modeled through
Eckert's reference enthalpy method 
\cite{
  eckert_engineering_relations_for_heat_transfer_and_friction_in_high_velocity_laminar_and_turbulent_flow}
and \nth{3} order piston theory
\cite{
meijer_generalized_formulation_and_review_of_piston_theory_for_airfoils}.
It was found that by including plasticity, 
the behavior of the panel changed after four loading cycles 
as compared to an otherwise equivalent linearly elastic panel;
each cycle represented one ten second flight.

This study investigates the role creep and plasticity play 
in the permanent deformation of a panel under repeated high speed flight.
The amount of deformation of a body undergoes due to creep increases 
with both time and temperature
\cite{roylance_mechanics_of_materials_text}. 
Plasticity through a von Mises yield surface is also modeled;
a multiplicatively decomposed version is used to allow for 
possible large strains.
The details of the model and implementation used are
discussed in 
\cite{ li_simulation_of_finite_strain_inelastic_phenomena_governed_by_creep_and_plasticity}.
\newline
\newline
Note to the reviewers: A more extensive review of the 
state of the art will be included in the 
final manuscript.

\section{Methods} \label{sec_methods}
% outline:
%  - Panel geometry used
%  - Explain Creep + Plasticity from Zhen's model
%  - Loading: Repeated historical temperature
%  - Compare results to literature

The goal of this study is to model the large creep deformation a canonical acreage
panel due to the thermal loads experienced from repeated high speed flight.


% \section{Discussion and Closing Remarks} \label{sec_closing_remarks}
% discussion text

% \section{Future Work} \label{sec_future}
% future work text
% 
% \section*{Acknowledgments}
% The authors thank the DoD Science, Mathematics, And 
% Research for Transformation (SMART) Scholarship
% for Service Program which sponsored this work.

\bibliography{monolith.bib}

\end{document}
