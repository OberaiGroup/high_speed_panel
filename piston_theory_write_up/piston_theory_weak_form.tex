\documentclass[a4paper, 12pt]{article}
\author{Justin L. Clough}
\title{Weak Formulation for \nth{2} Order Piston Theory}
\usepackage[margin=1in]{geometry}
\usepackage{float}
\usepackage{subfigure}
\usepackage[justification=centering]{caption}
\usepackage{enumerate}
\usepackage{multirow}
\usepackage{listings}
\lstset{
    escapechar=`,
    language=C++,
    numbers=left,
    tabsize=2,
    prebreak=\raisebox{0ex}[0ex][0ex]{\ensuremath{\hookleftarrow}},
    frame=single,
    breaklines=true,
}
\usepackage{graphicx}
\graphicspath{ {./} }
\usepackage{nameref}
\usepackage{amsmath}
\usepackage{amssymb}
\usepackage{amsfonts}
\usepackage[linesnumbered,ruled]{algorithm2e}
\usepackage{tikz}
\usetikzlibrary{calc,patterns,decorations.pathmorphing,decorations.markings,positioning,automata}
\usepackage{pgfplots}
\pgfplotsset{compat=1.5}
\usepackage{pgfplotstable}
\usepackage{makecell}
\usepackage{verbatim}
\usepackage[super]{nth}
\usepackage{physics}

\begin{document}
\maketitle

The interaction between a deformable 3D body and
a flow field is given below.
The flow field is modeled as interacting with the surface 
of the body through \nth{2} order piston theory.
The relation between stress and displacement in the body
is left generic so any model can be accommodated.


From \nth{2} order piston theory, the pressure acting
on a deformable surface in a flow is 
given by:

\begin{equation}
P = \frac{2q}{M_\infty} 
  \left[
    \frac{1}{\Vert V\Vert_{L_2}} \pdv{w}{t} + \theta 
    + \frac{ (\gamma+1) M_\infty}{4} 
      \left(
        \frac{1}{\Vert V\Vert_{L_2}} \pdv{w}{t} + \theta
      \right)^2
  \right]
  + P_\infty
  \label{eq_strong_piston}
\end{equation}

\noindent
where 
$P$ is the absolute pressure acting on a surface,
$q$ is the dynamic pressure of the flow, 
$M_\infty$ is the Mach number of the flow,
$V$ is the velocity of the flow,
$\gamma$ is the ratio of specific heats for the fluid,
$w$ is the deflection of the surface into the flow,
$t$ is time, 
$\theta$ is the angle of the deflection along the flows path,
and
$P_\infty$ is the far-field pressure of the fluid.
The dynamic pressure is:

\begin{equation}
q = \frac{ \rho_f \Vert V\Vert_{L_2}^2}{2}
\end{equation}

\noindent
where 
$\rho_f$ is the density of the fluid.
Expanding Equation \eqref{eq_strong_piston},
letting $\pdv{w}{t}=w,_t$,
and replacing constants with $a_i$ gives:

\begin{align}
P& 
  = \frac{2q}{M_\infty \Vert V\Vert_{L_2}} w,_t
  + \frac{2q}{M_\infty} \theta
  + \frac{(\gamma +1)q}{2\Vert V\Vert_{L_2}^2} (w,_t)^2
  + \frac{(\gamma+1)q}{\Vert V\Vert^2_{L_2}} w,_t \theta 
  + \frac{(\gamma+1)q}{2} \theta^2
  + P_\infty \\
P& 
  = a_1 w,_t
  + a_2 \theta
  + a_3 w,_t^2
  + a_4 w,_t \theta 
  + a_5 \theta^2
  + a_6
\end{align}

\noindent
It is assumed that the values of $a_i$ are given 
or can be calculated from given information about the flow.

The deformable body is a 3D region $\Omega$ with boundaries
$\Gamma_g$ and $\Gamma_h$.
The boundaries have an outward unit normal vector $n$.
Note that $\Gamma_g \cup \Gamma_h = \partial \Omega$
and the boundaries do not intersect.
The displacement of a point is the vector $u$.
The PDE and boundary conditions governing the displacement of the structure are:

\begin{align}
\nabla \cdot \sigma(u) &= f \text{ in $\Omega$} \\
u &= g \text{ on $\Gamma_g$} \\
\sigma n &= h \text{ on $\Gamma_h$}  \label{eq_nuemann}
\end{align}

\noindent
It is assumed that the stress-displacement relations, $\sigma (u)$, is known
as well as the values of the vectors $f$, $g$, and the material density $\rho$.
A boundary $\Gamma_p \subset \Gamma_h$ is treated as the 
surface interacting with the flow. 
The remaining surface is $\Gamma_{h-p} = \Gamma_h \setminus \Gamma_p$.
It is also assumed that boundary data $h$ is given on $\Gamma_{h-p}$
which represents all other prescribed surface tractions.
For the flow surface, the boundary condition in Equation \eqref{eq_nuemann}
is then recognized as $\sigma n = P n$.

The typical steps are then taken to construct a weak form
of the structure problem
by integrating over the domain after taking the 
dot product with a test function $v$.

\begin{equation}
\int_\Omega (v \cdot \nabla \cdot \sigma) d\Omega = \int_\Omega v \cdot  f d\Omega
\end{equation}

Integrating by parts, separating the boundary integral over $\Gamma_h$
into $\Gamma_p$ and $\Gamma_{h-p}$, and using Einstein notation give

\begin{equation}
\int  v_i,_j \sigma_{ij} d\Omega = \int_\Omega v_i f_i d\Omega
  +\int_{\Gamma_{h-p}} v_i h_i d\Gamma
  +\int_{\Gamma_p} v_i n_i P d\Gamma
\end{equation}

The right most term represents the pressure.
A stream direction, $s$, tangent to the surface 
is then defined as the unit vector of the flow velocity:
\begin{equation}
s = \frac{V}{\Vert V\Vert_{L_2}}
\end{equation}

\noindent 
The slope of deflection into the flow and along the stream direction,
$\theta$ in Equation \eqref{eq_strong_piston}, can then be 
defined as:

\begin{equation}
\theta = \left\Vert\pdv{w}{s}\right\Vert_{L_2}
\end{equation}


\end{document}
