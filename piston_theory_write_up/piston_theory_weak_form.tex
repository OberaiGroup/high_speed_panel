\documentclass[a4paper, 12pt]{article}
\author{Justin L. Clough}
\title{Weak Formulation for \nth{2} Order Piston Theory}
\usepackage[margin=1in]{geometry}
\usepackage{float}
\usepackage{subfigure}
\usepackage[justification=centering]{caption}
\usepackage{enumerate}
\usepackage{multirow}
\usepackage{listings}
\lstset{
    escapechar=`,
    language=C++,
    numbers=left,
    tabsize=2,
    prebreak=\raisebox{0ex}[0ex][0ex]{\ensuremath{\hookleftarrow}},
    frame=single,
    breaklines=true,
}
\usepackage{graphicx}
\graphicspath{ {./} }
\usepackage{nameref}
\usepackage{amsmath}
\usepackage{amssymb}
\usepackage{amsfonts}
\usepackage[linesnumbered,ruled]{algorithm2e}
\usepackage{tikz}
\usetikzlibrary{calc,patterns,decorations.pathmorphing,decorations.markings,positioning,automata}
\usepackage{pgfplots}
\pgfplotsset{compat=1.5}
\usepackage{pgfplotstable}
\usepackage{makecell}
\usepackage{verbatim}
\usepackage[super]{nth}
\usepackage{physics}

\begin{document}
\maketitle

From \nth{2} order piston theory, the pressure acting
on a deformable surface in a flow is 
given by:

\begin{equation}
P = \frac{2q}{M_\infty} 
  \left[
    \frac{1}{V} \pdv{w}{t} + \theta 
    + \frac{ (\gamma+1) M_\infty}{4} 
      \left(
        \frac{1}{V} \pdv{w}{t} + \theta
      \right)^2
  \right]
  + P_\infty
\end{equation}

\noindent
where 
$P$ is the absolute pressure acting on a surface,
$q$ is the dynamic pressure of the flow, 
$M_\infty$ is the Mach number of the flow,
$V$ is the speed of the flow,
$\gamma$ is the ratio of specific heats for the fluid,
$w$ is the deflection of the surface into the flow,
$t$ is time, 
$\theta$ is the angle of the deflection along the flow,
and
$P_\infty$ is the far-field pressure of the fluid.

\end{document}
