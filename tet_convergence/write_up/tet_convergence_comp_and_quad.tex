\documentclass[a4paper, 12pt]{article}
\author{Justin L. Clough}
\title{Convergence Study for Tetrahedrons with Composite and Quadratic Shape Functions in Bending}
\usepackage[margin=1in]{geometry}
\usepackage{float}
\usepackage{subfigure}
\usepackage[justification=centering]{caption}
\usepackage{enumerate}
\usepackage{multirow}
\usepackage{listings}
\lstset{
    escapechar=`,
    language=C++,
    numbers=left,
    tabsize=2,
    prebreak=\raisebox{0ex}[0ex][0ex]{\ensuremath{\hookleftarrow}},
    frame=single,
    breaklines=true,
}
\usepackage{graphicx}
\graphicspath{ {./} }
\usepackage{nameref}
\usepackage{amsmath}
\usepackage{amssymb}
\usepackage{amsfonts}
\usepackage[linesnumbered,ruled]{algorithm2e}
\usepackage{tikz}
\usetikzlibrary{calc,patterns,decorations.pathmorphing,decorations.markings,positioning,automata}
\usepackage{pgfplots}
\pgfplotsset{compat=1.5}
\usepackage{pgfplotstable}
\usepackage{makecell}
\usepackage{verbatim}
\usepackage[super]{nth}

\begin{document}
\maketitle

\begin{abstract}
The convergence of tetrahedral elements with
\nth{2} order shape functions are compared to those with composite shape functions
in a beam bending problem.
A linear elastic material model was used
and no time components were considered.
The geometry was a beam with a square cross-section and 
aspect ratio of 100.
The beam was cantilevered on one end and received a 
shearing traction at the other. 
The tip deflection of the finite element solution was
compared to that from Euler-Bernoulli beam theory. 
Tetrahedrons with \nth{2} order shape functions under-predicted
the theoretical solution by 5\% with 4 elements 
through the thickness of the beam;
elements with composite shape functions under-predicted 
the theoretical solution by 68\% with 32 elements through the thickness.
\end{abstract}

The performance of tetrahedron element with \nth{2} order shape functions
was compared to that of elements with composite shape functions 
(discussed in \cite{bib:composite_tet})
in a beam-bending problem.
The mesh was refined such that the aspect ratio of the elements
stayed at 10 as the number of elements through the thickness was increased.
The theoretical deflection from Euler-Bernoulli beam theory was used to
compared the finite element results.

The problem geometry was a beam with a square cross-section.
The cross-section was 0.01 m $ \times $ 0.01 m; 
the beam was 1 m long in the $Y$ direction.
The other edges of the beam were aligned with the $X$ and $Z$ axes.

A linear elastic material model was used. 
Young's Modulus was 4000 Pa;
Poisson's ratio was 0.25.
No time components were considered
and only the static solution was calculated.

The square face with normal in the negative $Y$ direction had 
all degrees of freedom fixed to act as a cantilever support. 
The face on the other end of the beam received a 
traction of 0.01 $\frac{N}{m^2}$ in the $Z$ direction.
From Euler-Bernoulli beam theory, the tip deflection
is modeled as:

\begin{equation}
  \delta = \frac{F L^3}{3 E I}
  \label{eq:EB_deflection}
\end{equation}

\noindent
where 
$F$ is the total force acting at the tip of the beam, 
$L$ is the length of the beam,
$E$ is Young's Modulus for the beam's material,
and
$I$ is the second moment of the cross-sectional area.
Evaluating Equation (\ref{eq:EB_deflection}) for the
given parameters of the problem, 
the theoretical tip deflection is 0.1 m.

A mesh was constructed to approximate the beam geometry
and used only tetrahedral elements.
The convergence of 
tetrahedrons with \nth{2} order shape functions (quadratic tet10 elements)
and separately with composite shape functions 
were measured.
An aspect ratio of 10 was used for all meshes
and the number of elements through the thickness of the beam was increased.
A direct linear algebra solver was used to solve the system.

\begin{figure}[H]
  \centering
    \begin{tikzpicture}
      \begin{axis}[
        legend pos=outer north east,
        grid=both,
        grid style={line width=.1pt, draw=gray!10},
        major grid style={line width=.2pt,draw=gray!50},
        xtick={},
        ytick={},
        minor tick num=5,
        title=Convergence of Tip Deflection,
        xlabel=Elements Through Thickness,
        ylabel={Tip Deflection [m]} ]
        \addplot table {quadratic_data_beam.txt};
          \addlegendentry{Quadratic}
        \addplot table {composite_data_beam.txt};
          \addlegendentry{Composite}
      \end{axis}
    \end{tikzpicture}
  \caption{Tip deflection for beam geometry with composite and quadratic elements.
           Analytical solution is 0.1 m.}
  \label{fig:label}
\end{figure}

The quadratic tet10 elements underestimated the tip deflection 
within 5\% of the analytical solution with 4 elements through
the thickness. 
This increased to 0.1\% with 16 elements through the thickness.
The composite tet10 underestimated the analytical tip deflection
by 68\% with 32 elements through the thickness.

\newpage
\begin{thebibliography}{99}

\bibitem{bib:composite_tet}
J. T. Ostien,
J. W. Foulk,
A. Mota,
M. G. Veilleux.
A 10-node Composite tetrahedral finite element for solid mechanics.
International Journal For Numerical Methods in Engineering.
2016:107:1145-1170. 

\end{thebibliography}


\end{document}
