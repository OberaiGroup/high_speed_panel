\documentclass[a4paper, 12pt]{article}
\author{Justin L. Clough}
\title{Convergence Study for Tetrahedrons with Composite and Quadratic Shape Functions in Bending}
\usepackage[margin=1in]{geometry}
\usepackage{float}
\usepackage{subfigure}
\usepackage[justification=centering]{caption}
\usepackage{enumerate}
\usepackage{multirow}
\usepackage{listings}
\lstset{
    escapechar=`,
    language=C++,
    numbers=left,
    tabsize=2,
    prebreak=\raisebox{0ex}[0ex][0ex]{\ensuremath{\hookleftarrow}},
    frame=single,
    breaklines=true,
}
\usepackage{graphicx}
\graphicspath{ {./} }
\usepackage{nameref}
\usepackage{amsmath}
\usepackage{amssymb}
\usepackage{amsfonts}
\usepackage[linesnumbered,ruled]{algorithm2e}
\usepackage{tikz}
\usetikzlibrary{calc,patterns,decorations.pathmorphing,decorations.markings,positioning,automata}
\usepackage{pgfplots}
\pgfplotsset{compat=1.5}
\usepackage{pgfplotstable}
\usepackage{makecell}
\usepackage{verbatim}

\begin{document}
\maketitle

\begin{abstract}
Is this the abstract text? yep.
\end{abstract}

Intro statement here.

The problem geometry was a beam with a square cross-section.
The cross-section was 0.01 m $ \times $ 0.01 m; 
the beam was 1 m long in the $Y$ direction.
The other edges of the beam were aligned with the $X$ and $Z$ axes.

A linear elastic material model was used. 
Young's Modulus was 4000 Pa.
Poisson's ratio was 0.25.

The square face with normal in the negative $Y$ direction had 
all degrees of freedom fixed to act as a cantilever support. 
The face on the other end of the beam received a 
traction of 0.01 $\frac{N}{m^2}$ in the $Z$ direction.
From Euler-Bernoulli beam theory, the tip deflection
is modeled as:

\begin{equation}
  \delta = \frac{F L^3}{3 E I}
  \label{eq:EB_deflection}
\end{equation}

\noindent
where 
$F$ is the total force acting at the tip of the beam, 
$L$ is the length of the beam,
$E$ is Young's Modulus for the beam's material,
and
$I$ is the second moment of the cross-sectional area.
Evaluating Equation (\ref{eq:EB_deflection}) for the
given parameters of the problem, 
the theoretical tip deflection is 0.1 m.

\begin{figure}[H]
  \centering
    \begin{tikzpicture}
      \begin{axis}[
        legend pos=outer north east,
        grid=both,
        grid style={line width=.1pt, draw=gray!10},
        major grid style={line width=.2pt,draw=gray!50},
        xtick={},
        ytick={},
        minor tick num=5,
        title=Convergence of Tip Deflection,
        xlabel=Elements Through Thickness,
        ylabel={Tip Deflection [m]} ]
        \addplot table {composite_data_beam.txt};
          \addlegendentry{Composite}
        \addplot table {quadratic_data_beam.txt};
          \addlegendentry{Quadratic}
      \end{axis}
    \end{tikzpicture}
  \caption{Tip deflection for beam geometry with composite and quadratic elements.
           Analytical solution is 0.1 m.}
  \label{fig:label}
\end{figure}

more text here!

\newpage
\begin{thebibliography}{99}

\bibitem{bib:composite_tet}
J. T. Ostien,
J. W. Foulk,
A. Mota,
M. G. Veilleux.
A 10-node Composite tetrahedral finite element for solid mechanics.
International Journal For Numerical Methods in Engineering.
2016:107:1145-1170. 

\end{thebibliography}


\end{document}
